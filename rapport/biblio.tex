\documentclass[12 pt]{article}
\usepackage{inputenc}
\usepackage [french]{babel}
\bibliographystyle{alpha}
\begin{document}
\section{Elements bibliographiques}
L'article \cite{Ufuk} propose une m$\acute{e}$thode d'impl$\acute{e}$mentation de mesure de rythme cardiaque $\grave{a}$ partir d'enregistrements vid$\acute{e}$os. 
A un instant donn$\acute{e}$, on d$\acute{e}$tecte le visage de l'individu, puis la peau gr$\hat{a}$ce au flux YCbCr. On obtient les signaux PPG en utilisant deux vecteurs de l'espace RGB. 
Apr$\acute{e}$s avoir trait$\acute{e}$ les signaux (filtre passe-bande, transformée de Fourier, …) de chaque fen$\hat{e}$tre de la vid$\acute{e}$o, on obtient la pulsation cardiaque.
\bibliography{test}
\end{document}