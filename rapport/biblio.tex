\documentclass[12 pt]{article}
\usepackage{inputenc}
\usepackage [french]{babel}
\bibliographystyle{alpha}
\begin{document}
\section{Elements bibliographiques}
\begin{itemize}
\item L'article \cite{Patel14} Heart attack detection and Medical attention using Motion Sensing Device-Kinect explique les recherches sur les attaques cardiaques avec une Kinect. Cette article peut être utile car pour repérer une attaque cardiaque les chercheurs mesures le rythme cardiaque du patient. L'article décrit l'utilisation de la Kinect pour arriver à leur fin.
\item L'article \cite{Ufuk} propose une m$\acute{e}$thode d'impl$\acute{e}$mentation de mesure de rythme cardiaque $\grave{a}$ partir d'enregistrements vid$\acute{e}$os.
A un instant donn$\acute{e}$, on d$\acute{e}$tecte le visage de l'individu, puis la peau gr$\hat{a}$ce au flux YCbCr. On obtient les signaux PPG en utilisant deux vecteurs de l'espace RGB. 
Apr$\acute{e}$s avoir trait$\acute{e}$ les signaux (filtre passe-bande, transformée de Fourier, …) de chaque fen$\hat{e}$tre de la vid$\acute{e}$o, on obtient la pulsation cardiaque.

\end{itemize}

\bibliography{biblio}
\end{document}